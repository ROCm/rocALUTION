\chapter {Plug-ins}

Note that there is no MPI support for the plug-ins.

\section{FORTRAN}

PARALUTION comes with an easy to use Fortran plug-in.
Currently it supports \emph{COO} and \emph{CSR} input matrix formats and uses the intrinsic \emph{ISO\_BIND\_C} to transfer data between Fortran and PARALUTION.
The argument passing for the \emph{COO} and \emph{CSR} subroutine calls only differ in the matrix related arrays.

\lstinputlisting[title="Example of Fortran subroutine call using COO matrix format"]{./src/Fortran_COO.txt}

\lstinputlisting[title="Example of Fortran subroutine call using CSR matrix format"]{./src/Fortran_CSR.txt}

The arguments include:

\begin{itemize}
\itemsep0em
\item (2) Number of rows, number of columns, number of non-zero elements
\item (3) Solver: CG, BiCGStab, GMRES, Fixed-Point
\item (4) Operator matrix format: DENSE, CSR, MCSR, COO, DIA, ELL, HYB
\item (5) Preconditioner: None, Jacobi, MultiColoredGS, MultiColoredSGS, ILU, MultiColoredILU
\item (6) Preconditioner matrix format: DENSE, CSR, MCSR, COO, DIA, ELL, HYB
\item (7) Row index (COO) or row offset pointer (CSR), column index, right-hand side
\item (8) Absolute tolerance, relative tolerance, divergence tolerance, maximum number of iterations
\item (9) Size of the Krylov subspace (GMRES), ILU(p), ILU(q)
\item (10) Outputs: solution vector, number of iterations needed to converge, final residual norm, status code
\end{itemize}

A detailed listing is also given in the header of the PARALUTION Fortran plug-in file.
\\
For a successful integration of PARALUTION into Fortran code a compiled PARALUTION library is necessary.
Furthermore, you need to compile and link the Fortran plug-in (located in \emph{src/plug-ins}) because it is not included in the compiled library file.
To achieve this, a simple Makefile can be used.

\lstinputlisting[title="Example Makefile for PARALUTION integration to Fortran"]{./src/Fortran_Makefile.txt}

\textbf{\emph{Note}} Examples are in \emph{src/examples/fortran}.
