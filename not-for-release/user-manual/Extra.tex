\chapter{Advanced Techniques}

\section{Memory Allocation}

All data which is passed to and from PARALUTION (via SetDataPtr/LeaveDataPtr) is using the memory handling functions described in the code. By default, the library uses standard \emph{new} and \emph{delete} functions for the host data.

\begin{table}[H]
\begin{tabular}{l}
Available \\ \hline
B,S,M    
\end{tabular}
\end{table}

\subsection{Allocation Problems}

If the allocation fails, the library will report an error and exits. If the user requires a special treatment, it has to be placed in the file \emph{src/utils/allocate\_free.cpp}.

\begin{table}[H]
\begin{tabular}{l}
Available \\ \hline
B,S,M    
\end{tabular}
\end{table}

\subsection{Memory Alignment}

The library can also handle special memory alignment functions. This feature need to be uncommented before the compilation process in the file \emph{src/utils/allocate\_free.cpp}.


\subsection{Pinned Memory Allocation (CUDA)}

By default, the standard host memory allocation is realized by \emph{new} and \emph{delete}. For a better PCI-E transfers for NVIDIA GPUs, the user can also use pinned host memory. This can be activated by uncommenting the corresponding macro in \emph{src/utils/allocate\_free.hpp}.
\begin{table}[H]
\begin{tabular}{l}
Available \\ \hline
B,S,M    
\end{tabular}
\end{table}
